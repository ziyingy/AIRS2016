\begin{abstract}
We explore the implications of tied scores arising in the document
similarity scoring regimes that are used when queries are processed
in a retrieval engine.
Our investigation has two parts: first, we evaluate past TREC runs to
determine the prevalence and impact of tied scores, to understand the
alternative treatments that might be used to handle them; and second,
we explore the implications of what might be thought of as
``deliberate'' tied scores, in order to allow for faster search.
In the first part of our investigation we show that while tied scores
had the potential to be relatively disruptive to TREC evaluations, in
practice their effect was relatively minor.
The second part of our exploration helps understand why that was so,
and shows that quite marked levels of score rounding can be
tolerated, without greatly affecting the ability to compare between
systems.
The latter finding offers the potential for approximate scoring
regimes that provide faster query processing with little or no loss
of effectiveness.

\keywords{Information retrieval, effectiveness, document scoring,
ties}
\end{abstract}
