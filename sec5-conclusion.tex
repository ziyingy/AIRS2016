\section{Conclusion and Future Work}
\label{sec-conclusion}

We have explored the impact of score ties on the evaluation of
retrieval system effectiveness.
Ties have the potential to affect system comparisons, and using TREC
data, we showed that a small number of systems did indeed generate
runs with very ambiguous score outcomes, but that -- fortunately --
the overall conclusions from those rounds of experimentation were
unlikely to have been compromised.
We further demonstrated that allowing a controlled grouping of scores
in runs -- in a sense, permitting the deliberate introduction of ties
-- resulted in only small changes in the ability to compare systems.
This approach represents a novel direction in which retrieval
efficiency improvements might be achieved.
We have not yet addressed the question of how those efficiency gains
might be achieved, and a clear direction for future work is to
reexamine the computation embedded in standard similarity scoring
regimes and existing dynamic pruning heuristics, to identify and
measure ways in which processing economies might accrue through the
use of inexact scoring.


