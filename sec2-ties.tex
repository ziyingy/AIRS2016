\section{Ties, and Methods for Dealing With Them}
\label{sec-ties}

\myparagraph{Terminology}

\begin{figure}[t]
\centering
\newcommand{\tabent}[1]{\makebox[18mm][l]{#1}}
\begin{tabular}{l|@{\hspace{1.0em}} *{10}{l@{\hspace{1.8em}}}}
\tabent{rank, $k$}
	& 1
	    & 2
	    	& 3
		    & 4
		    	& 5
		            & 6
			    	& 7
			    	    & 8
				        & 9
					    & 10
\\
\hline
\tabent{document, $d_k$}
	& D
	    & H
	        & A
		    & C
		        & M
			    & S
			        & W
				    & B
				        & E
					    & J
\\
\tabent{gain, $r_k$}
	& 0
	    & 0
	        & 1
		    & 1
		        & 0
			    & 1
			        & 1
				    & 0
				        & 0
					    & 1
\\
\tabent{score}
	& 9.8
	    & 9.3
	        & 9.3
		    & 9.3
		        & 8.4
			    & 8.4
			        & 8.2
				    & 8.0
				        & 8.0
					    & 8.0
\\
\tabent{groups}
	& \multicolumn{1}{@{}l}{$b_1{=}1$}
	    & \multicolumn{1}{@{}l}{$b_2{=}2$}
	        &
		    &
		        & \multicolumn{1}{@{}l}{$b_3{=}5$}
			    &
			        & \multicolumn{1}{@{}l}{$b_4{=}7$}
				    & \multicolumn{1}{@{}l}{$b_5{=}8$}
				        &
				            &
\\
\end{tabular}

\caption{Example run showing five equi-score groups.
\label{fig-example}}
\end{figure}

We suppose that the similarity scores generated for a query divide
the document ranking -- the {\emph{run}} -- into groups in which all
of the documents have the same score.
Let $b_i$ be the rank in the run at which the $i$\,th equi-score
group commences, with, by definition, $b_1=1$.
That is, the $i$\,th group of tied dcouments spans the items
$[b_i\ldots b_{i+1}-1]$, and contains $b_{i+1}-b_i$ documents.
For example, consider the run with scores shown in
Figure~\ref{fig-example}.
In this example, the top-$10$ document ranking (with documemts
labeled consecutively for convenience of reference) is assumed to
contain five different computed similarity scores.
The last row shows a presumed relevance value for each
corresponding document, with ``$0$'' indicating not relevant and
``$1$'' indicating relevant.
If the scores are ignored and only the list of relevance values is
available, computation of the metric precision at depth $k=5$ (P@$5$)
yields a score of $2/5=0.4$, because there are two ``$1$''s
among the first five gain values.
Similarly, the ranking has a reciprocal rank (RR) score of
$1/3=0.333$, since the first relevant document appears at rank $k=3$.
Other metrics such as average precision (AP), rank-biased precision
(RBP), and normalized discounted cumulative gain (NDCG), can also be
computed, based solely on the ``gain'' row, without consideration of
the document labels or their actual scores.

When scores are include in the evaluation, the situation changes.
Now documents E and F in the example have the same similarity score,
and are part of the third tied group.
That means that P@$5$ might be either $2/5$ or $3/5$, depending on
the tie-breaking rule employed to order E and F.
Similarly, RR might be $1/2$ or $1/3$, because of the tie between B
and C and D, but RR cannot be $1/4$.


\myparagraph{Run Order}

{\todo{could use the ordering generated by the system, whatever it is, and just say, we presume that they knew what they were doing}}

\myparagraph{External Tie-Break Rule}

\todo{{\tt{trec\_eval}}, and similar}

\myparagraph{Averaging Across Permutations}

{\citet{mn08ecir}, and describe basis for their formulations}

\myparagraph{Limits}

{\todo{optimistic and pessimistic, to get bounds}}
